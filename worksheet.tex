\documentclass{dcbl/challenge}

\setdoctitle{Collaborate using GitHub}
\setdocauthor{Stephan Bökelmann}
\setdocemail{sboekelmann@ep1.rub.de}
\setdocinstitute{AG Physik der Hadronen und Kerne}


\begin{document}

The most useful features of GitHub and similar platforms are not only in storing your code, but also in sharing it.
There are certain rules, we should know, in order for us to be effective collaborators in GitHub.
Even though there are several different modes of collaborating, the one explained here is usually regarded as the most basic, yet effective one.
Our goal is to work with multiple people on the same project, therefore you should work with at least one partner in this exercise.

\section*{Exercises}
\begin{aufgabe}
    Go to \url{https://github.com/} and create a new repository.
    Initialize this repository with a \texttt{README.md} file.
    Clone this repository to your computer, change the \texttt{README.md} file and push the changes to your repository.
    This should be done by all partners in this exercise.
\end{aufgabe}

\begin{aufgabe}
    Exchange the links to your GitHub repository with your partner.
    You should be able to see your partners project. 
    Even though you could now just clone the repository and use it, you would work differently if you wanted to contribute to it.
    See the button \textbf{Fork} in the top right corner?
    Click it to create your own instance of the repository in your own GitHub account.
    All partners should fork a repository of one other partner.
\end{aufgabe}

\begin{aufgabe}
    Clone the forked repository from your own account on github. 
    Make some changes to the \texttt{README.md} file and push the changes to your repository.
    After pushing the changes, inspect the GitHub-webpage of your own fork.
\end{aufgabe}

\begin{aufgabe}
    The changes you made are now still local to your own fork, but we'd like to upstream the changes we've made. 
    This is typically done, if we believe, that our changes are a relevant contribution to the project.
    In order to do this, we need to create a pull request.
    This should be done by all partners in this exercise.
    Creating a pull-request opens a new webpage.
    Use this page, to tell the partner about the recent changes, and why you believe, that this change is relevant.
    Your partner will get a notification about the pull-request in their repository. 
\end{aufgabe}

\begin{abstract}
    Go back to your own repository, that your partner forked. 
    Look at the pull-request tab on the top. 
    Go into the pull-request view, review the changes and if you are fine with them, merge them into your own repository!
\end{abstract}

\section*{Annotations}
\begin{enumerate}
    \item Link zu einem YouTube-Video: \url{https://www.youtube.com}
\end{enumerate}

\end{document}
